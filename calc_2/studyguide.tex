\documentclass{article}
\usepackage[margin=0.5in]{geometry}
\usepackage{hyperref}
\usepackage{amsmath}
\usepackage[dvipsnames]{xcolor}
\begin{document}
    \section{Calc 2 Basics}
        \color{BrickRed}
        \subsection*{5.3 Average Value}
            Integrals are pretty much an infinite amount of Riemann sums.
            If $f$ is integrable on $[a, b]$, then its average value (or mean) on $[a, b]$ is $\frac{1}{b-a}\int_{a}^{b}f(x)dx$.
        \color{Black}
        \subsection*{5.5 U-Substitution}
        \subsection*{5.6 }
        \subsection*{6.1 }
        \subsection*{6.2 }
    \section{Integral Applications and Sections}
        \subsection*{8.2}
        \subsection*{8.3}
        \subsection*{8.4}
        \subsection*{8.5}
    \section{Sequences, Series, and Tests}
        \subsection*{8.8}
        \subsection*{10.1}
        \subsection*{10.2}
        \subsection*{10.3}
        \subsection*{10.4}
    \section{Sequences, Series, and Tests cont.}
        \subsection*{10.5}
        \subsection*{10.6}
        \subsection*{10.7}
        \subsection*{10.8}
        \subsection*{10.9}
    \section{Parametric Curves}
        \subsection*{10.10}
        \color{OliveGreen}
        \subsection*{6.3 Arc Length}
            Pythagorean's Theoreom can be applied to find the length of a segment \(f(x)\).
            If \(ds\) is equal to a single straight segment in \(f(x)\), then  \(dx\) is equal to the horizontal length and \(dy\) is equal to its vertical length. 
            \begin{equation}
                \begin{split}
                    (ds)^2 = (dx)^2 + (dy)^2 \\
                    \sqrt{(ds)^2} = \sqrt{(dx)^2 + (dy)^2}\\
                    ds = dx  \sqrt{(dx)^2 / (dx)^2 + (dy)^2 / (dx) ^2}\\
                    ds = \sqrt{1 + (dy/dx)^2}dx\\
                \end{split}
            \end{equation}
            By taking the integral of this, you can get the total length of the segment.
            \begin{equation}
                s = \int_a^b \sqrt{1 + (f(x))^2} \, dx
            \end{equation}
        \color{Black}
        \subsection*{6.4}
        \subsection*{11.1}
        \subsection*{11.2}
    \section{Polar Coordinates}
        \subsection*{11.3}
        \subsection*{11.4}
        \subsection*{11.5}
\end{document}
