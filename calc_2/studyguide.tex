\documentclass{article}
\title{Calc 2 Complete Study Guide}
\usepackage[margin=0.5in]{geometry}
\usepackage{hyperref}
\usepackage{amsmath}
\usepackage[dvipsnames]{xcolor}
\begin{document}
    \section{Integral Basics}
        \color{BrickRed}
        \subsection*{5.3 Average Value}
            Integrals are pretty much an infinite amount of Riemann sums.
            If $f$ is integrable on $[a, b]$, then its average value (or mean) on $[a, b]$ is $\frac{1}{b-a}\int_{a}^{b}f(x)dx$.
        \color{Violet}
        \subsection*{5.5 U-Substitution}
            Since $du$ can be substituted for $(\frac{du}{dx})dx$, u-substitution is possible.\\
            Ex. Find $\int (x^3 + x)^5(3x^2 + 1)dx$.
            \begin{equation}
                \begin{split}
                    \int_{a}^{b} (x^3 + x)^5(3x^2 + 1)dx\\
                    \int_{a}^{b} (u)^5(3x^2 + 1)dx\\
                    u = x^3 + x\\
                    \frac{du}{dx} = 3x^2 + 1 \\
                    \int_{a^3 + a}^{b^3 + b} (u)^5(\frac{du}{dx})dx\\
                    \int_{a^3 + a}^{b^3 + b} (u)^5du\\
                    \int_{a^3 + a}^{b^3 + b} (u)^5du\\
                    \frac{(b^3 + b)^6}{6} - \frac{(a^3 + a)^6}{6}
            \end{split}
                \end{equation}
        \color{NavyBlue}
        \subsection*{5.6 Area Between Curves}
            If $f(x) \geq g(x)$ for $[a, b]$, then the area between both curves from $a$ to $b$ is equal to $A = \int_{a}^{b} [f(x) - g(x)] dx$. 
        \color{ForestGreen}
        \subsection*{6.1 Volume Using Integrals: Disk and Washer Methods}
            The volume of a solid of integrable cross-sectional area $A(x)$ from $x=a$ to $x=b$ is equal to $\int_{a}^{b} A(x)dx$. \\
            The volume of a disk created by rotating a function $R(x)$ around $y=0$ (the x-axis) for the same bounds is equal to $\int_{a}^{b} \pi [R(x)]^2 dx$. \\
            The volume of a disk created by rotating a function $R(y)$ around $x=3$ for the same bounds is equal to $\int_{R((a)}^{R(b)} \pi [R(y)-3]^2 dy$. The reason 3 is subtracted from $R(y)$ is because the axis of rotation is closer to the area being rotated, so the radius of rotation will be less than if $x=0$.\\
            The volume of a washer created by rotating the space between $R(x)$ and $r(x)$ around $y=1$ is equal to $\int_{a}^{b} \pi ([R(x) - 1]^2 - [r(x) - 1]^2) dx$. The reason a 1 is subtracted is because the axis of rotation is closer to the area being rotated, so the radius of rotation will be less than if $y=0$. \\
            The volume of a washer created by rotating the space between R(y) and r(y) around x=0 (the y-axis) is equal to $\int_{R(a)}^{R(b)} \pi ([R(y)]^2 - [r(y)]^2) dx$. \\
        \color{Orange}
        \subsection*{6.2 Volumes Using Integrals: Shell Method}
            *Summarize Shell Method*
        \color{Black}
    \section{Integral Applications and Sections}
        \subsection*{8.2}
        \subsection*{8.3}
        \subsection*{8.4}
        \subsection*{8.5}
    \section{Sequences, Series, and Tests}
        \subsection*{8.8}
        \subsection*{10.1}
        \subsection*{10.2}
        \subsection*{10.3}
        \subsection*{10.4}
    \section{Sequences, Series, and Tests cont.}
        \subsection*{10.5}
        \subsection*{10.6}
        \subsection*{10.7}
        \subsection*{10.8}
        \subsection*{10.9}
    \section{Parametric Curves}
        \subsection*{10.10}
        \color{OliveGreen}
        \subsection*{6.3 Arc Length}
            Pythagorean's Theoreom can be applied to find the length of a segment \(f(x)\).
            If \(ds\) is equal to a single straight segment in \(f(x)\), then  \(dx\) is equal to the horizontal length and \(dy\) is equal to its vertical length. 
            \begin{equation}
                \begin{split}
                    (ds)^2 = (dx)^2 + (dy)^2 \\
                    \sqrt{(ds)^2} = \sqrt{(dx)^2 + (dy)^2}\\
                    ds = dx  \sqrt{(dx)^2 / (dx)^2 + (dy)^2 / (dx) ^2}\\
                    ds = \sqrt{1 + (dy/dx)^2}dx\\
                \end{split}
            \end{equation}
            By taking the integral of this, you can get the total length of the segment.
            \begin{equation}
                s = \int_a^b \sqrt{1 + (f(x))^2} \, dx
            \end{equation}
        \color{Black}
        \subsection*{6.4}
        \subsection*{11.1}
        \subsection*{11.2}
    \section{Polar Coordinates}
        \subsection*{11.3}
        \subsection*{11.4}
        \subsection*{11.5}
        \color{Aquamarine}
        \subsection*{Complex Numbers}
            Standard form = $a + bi$. $a$ is real portion, $bi$ is imaginary. 
            To graph: 3 units to the right, imaginary is 4 so 4 units up. To calculate absolute value, $/abs{a+bi} = /sqrt{a^2 + b^2}$\\
            For -5 + 12i: \\
            Graph 5 units to the left, and up 12 units. \\
            $\lvert {-5 + 12i}\rvert = \sqrt{(-5)^2 + 12^2}$ \\
            $\lvert {-5 + 12i}\rvert = \sqrt{(25 + 144}$ \\
            $\lvert {-5 + 12i}\rvert = \sqrt{169}$ \\
            $\lvert {-5 + 12i}\rvert = 13$ \\
            \\
            An example of how a square root of negative looks: $\sqrt{-80} =  i * \sqrt{16} * \sqrt{5} = 4i \sqrt{5}$
            \begin{equation}
                \begin{split}
                    i = \sqrt{-1} \\
                    i^2 = -1 \\
                    i^3 = -i \\
                    i^4 = 1 \\
                \end{split}
            \end{equation}
            Idea: Associate polar coordinates to complex numbers \\
            $z = x+iy$ \\
            $r = \lvert z \rvert = \sqrt{x^2 + y^2}$ \\
            if $x \geq 0$, $\theta = arctan \frac{y}{x}$ \\
            if $x \leq 0$, $\theta = arctan \frac{y}{x} + \pi$ \\
            if $x = 0, \theta = \pm \frac{\pi}{2}$ \\ 
            Find polar coordinates of $z = -2\sqrt{2} + 2\sqrt{2}i$\\
            $r = \lvert z \rvert = \sqrt{(-2\sqrt{2})^2  + (2\sqrt{2})^2} = \sqrt{16} = 4$ \\
            Since $x \leq 0$, $\theta =  arctan \frac{y}{x} + \pi = \frac{3\pi}{4} $ \\
            Polar Coordinates = $(4cos\frac{3\pi}{4},4sin\frac{3\pi}{4})$
\end{document}
